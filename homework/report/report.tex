\documentclass[UTF8]{article}
\usepackage{ctex}
\usepackage{hyperref}
\usepackage{setspace}
\usepackage{listings}
\usepackage{xcolor}
\usepackage{ulem}
\usepackage{amsmath}
\usepackage{amsthm}
\usepackage{amsfonts}
\usepackage{amssymb}
\usepackage{booktabs}
\usepackage{graphicx}
\usepackage{multicol}
\usepackage[top = 0.8in, bottom = 0.8in, left = 0.8in, right = 0.8in]{geometry}
\usepackage{cite}
\bibliographystyle{plain}

\newcommand \insertsubject {{车牌识别大作业报告}}

\hypersetup
{
	pdfauthor = 李沿橙 1900012766,
	pdftitle = \insertsubject,
	pdfsubject = \insertsubject,
	pdfkeywords = \insertsubject,
	colorlinks = true,
	linkcolor = black,
	anchorcolor = black,
	citecolor = black,
	urlcolor = black
}

\title{\insertsubject}
\author{李沿橙 1900012766}
\date{\today}

\begin{document}

	\maketitle

	\section{数据准备}

	\subsection{文件结构}

	下面的树形列表给出了原数据的结构,便于接下来对数据的加工:
	
	\begin{multicols}{2}
	\begin{itemize}
		\item Plate\_dataset/AC
		\begin{itemize}
			\item test
			\begin{itemize}
				\item jpeg
				\begin{itemize}
					\item 1.jpg
					\item 2.jpg
					\item ……
				\end{itemize}
				\item xml
				\begin{itemize}
					\item 1.xml
					\item 2.xml
					\item ……
				\end{itemize}
			\end{itemize}
			\item train
			\begin{itemize}
				\item jpeg
				\begin{itemize}
					\item 1.jpg
					\item 2.jpg
					\item ……
				\end{itemize}
				\item xml
				\begin{itemize}
					\item 1.xml
					\item 2.xml
					\item ……
				\end{itemize}
			\end{itemize}
		\end{itemize}
		\item Chars\_data
		\begin{itemize}
			\item 0
			\begin{itemize}
				\item *.jpg(文件名格式不固定)
			\end{itemize}
			\item ……
			\item A
			\item ……
		\end{itemize}
	\end{itemize}
	\end{multicols}

	\subsection{xml 文件}

	train 和 test 中的 xml 文件结构相同,里面包含了\cite{homework}:
	
	\begin{multicols}{2}
	\begin{itemize}
		\item 对应的图片文件名
		\item 长宽(总为 $352 \times 240$)、通道数(3)
		\item object 标签
		\begin{itemize}
			\item name(总为 plate)
			\item platetext(车牌号)
			\item bndbox(车牌的\textbf{矩形}范围)
			\begin{itemize}
				\item xmin
				\item ymin
				\item xmax
				\item ymax
			\end{itemize}
		\end{itemize}
	\end{itemize}
	\end{multicols}
	
	\subsection{将数据整合到 DataLoader}

	为了便于后续的训练,需要把以上数据放入 DataLoader。方法是:准备好数据后,使用 TensorDataset 合并图片和标签,再将这个 TensorDataset 作为参数传给 DataLoader。\cite{dataloader}

	\bibliography{ref}

\end{document}